%% LyX 1.6.7 created this file.  For more info, see http://www.lyx.org/.
%% Do not edit unless you really know what you are doing.
\documentclass[english]{article}
\usepackage[T1]{fontenc}
\usepackage[latin9]{inputenc}
\usepackage{amsbsy}
\usepackage{esint}
\usepackage{babel}

\begin{document}

\section*{In Flight Trajectory}

To derive the in flight trajectory of the ping pong ball, we subject
it to three forces: gravity, drag by air resistance and Magnus force.
The free body diagram of the ping pong ball is shown in Figure (??),
and we can use it to derive the differential equation governing the
trajectory. We can immediately consider the following equations,\[
\mathbf{F}=\mathbf{F}_{G}+\mathbf{F}_{M}+\mathbf{F}_{D}\]


\[
\mathbf{F}_{G}=-m\cdot g\mathbf{\hat{k}}\]
\[
\mathbf{F}_{D}=-\frac{1}{2}\rho C_{D}(\pi R)^{2}||\mathbf{v}||\mathbf{v}\]


where \textbf{$\mathbf{F}_{G}$}, \textbf{$\mathbf{F}_{D}$} are the
regular equations often used. We denote $m$ as the mass of the ball,
$g$ as the acceleration due to gravity, $\rho$ as the air density,
$C_{D}$ as the drag coefficient, and $R$ as the radius of the ball.
To derive the Magnus force takes a little more effort. Using the Kutta-Joukowski
Lift Theorem for a cylinder, we find that $\mathbf{F}_{M}=\rho(2\pi R)^{2}(\mathbf{\boldsymbol{\omega}}\times\mathbf{v})$
{[}1{]}. We can then just integrate over the sphere along the axis
of rotation, to find that $\mathbf{F}_{M}=\int_{-R}^{R}\rho(2\pi R)^{2}(\mathbf{\boldsymbol{\omega}}\times\mathbf{v})=\frac{16}{3}\pi^{2}\rho R^{3}(\mathbf{\boldsymbol{\omega}}\times\mathbf{v})$.
With that we find that\[
\mathbf{F}=m\frac{d^{2}r}{dt^{2}}=-m\cdot g\mathbf{\hat{k}+}\frac{16}{3}\pi^{2}\rho R^{3}(\mathbf{\boldsymbol{\omega}}\times\mathbf{v})-\frac{1}{2}\rho C_{D}(\pi R)^{2}||\mathbf{v}||\mathbf{v}\]


by taking the sum of the individual forces. We further assume that
the angular velocity is non-constant, and thus decreases over time
due to drag torque. We may use an analogous formula for the drag torque
found in {[}2{]} where $\boldsymbol{\tau}_{D}=-\frac{1}{2}C_{\omega}\rho R^{5}||\boldsymbol{\omega}||\boldsymbol{\omega}$.
When numerically solving these differential equations, we can also
adjust the rotational velocity in parallel to the main calculation.
\end{document}
